
%% bare_conf.tex
%% V1.3
%% 2007/01/11
%% by Michael Shell
%% See:
%% http://www.michaelshell.org/
%% for current contact information.
%%
%% This is a skeleton file demonstrating the use of IEEEtran.cls
%% (requires IEEEtran.cls version 1.7 or later) with an IEEE conference paper.
%%
%% Support sites:
%% http://www.michaelshell.org/tex/ieeetran/
%% http://www.ctan.org/tex-archive/macros/latex/contrib/IEEEtran/
%% and
%% http://www.ieee.org/

%%*************************************************************************
%% Legal Notice:
%% This code is offered as-is without any warranty either expressed or
%% implied; without even the implied warranty of MERCHANTABILITY or
%% FITNESS FOR A PARTICULAR PURPOSE!
%% User assumes all risk.
%% In no event shall IEEE or any contributor to this code be liable for
%% any damages or losses, including, but not limited to, incidental,
%% consequential, or any other damages, resulting from the use or misuse
%% of any information contained here.
%%
%% All comments are the opinions of their respective authors and are not
%% necessarily endorsed by the IEEE.
%%
%% This work is distributed under the LaTeX Project Public License (LPPL)
%% ( http://www.latex-project.org/ ) version 1.3, and may be freely used,
%% distributed and modified. A copy of the LPPL, version 1.3, is included
%% in the base LaTeX documentation of all distributions of LaTeX released
%% 2003/12/01 or later.
%% Retain all contribution notices and credits.
%% ** Modified files should be clearly indicated as such, including  **
%% ** renaming them and changing author support contact information. **
%%
%% File list of work: IEEEtran.cls, IEEEtran_HOWTO.pdf, bare_adv.tex,
%%                    bare_conf.tex, bare_jrnl.tex, bare_jrnl_compsoc.tex
%%*************************************************************************

% *** Authors should verify (and, if needed, correct) their LaTeX system  ***
% *** with the testflow diagnostic prior to trusting their LaTeX platform ***
% *** with production work. IEEE's font choices can trigger bugs that do  ***
% *** not appear when using other class files.                            ***
% The testflow support page is at:
% http://www.michaelshell.org/tex/testflow/



% Note that the a4paper option is mainly intended so that authors in
% countries using A4 can easily print to A4 and see how their papers will
% look in print - the typesetting of the document will not typically be
% affected with changes in paper size (but the bottom and side margins will).
% Use the testflow package mentioned above to verify correct handling of
% both paper sizes by the user's LaTeX system.
%
% Also note that the "draftcls" or "draftclsnofoot", not "draft", option
% should be used if it is desired that the figures are to be displayed in
% draft mode.
%
\documentclass[conference]{IEEEtran}

\IEEEoverridecommandlockouts
\usepackage{cite}
\usepackage{amsmath,amssymb,amsfonts}
\usepackage{algorithmic}
\usepackage{graphicx}
\usepackage{textcomp}
\usepackage{caption}
\usepackage{xcolor}
\def\BibTeX{{\rm B\kern-.05em{\sc i\kern-.025em b}\kern-.08em
    T\kern-.1667em\lower.7ex\hbox{E}\kern-.125emX}}

\RequirePackage[utf8]{inputenc}
% Add the compsoc option for Computer Society conferences.
%
% If IEEEtran.cls has not been installed into the LaTeX system files,
% manually specify the path to it like:
% \documentclass[conference]{../sty/IEEEtran}





% Some very useful LaTeX packages include:
% (uncomment the ones you want to load)


% *** MISC UTILITY PACKAGES ***
%
%\usepackage{ifpdf}
% Heiko Oberdiek's ifpdf.sty is very useful if you need conditional
% compilation based on whether the output is pdf or dvi.
% usage:
% \ifpdf
%   % pdf code
% \else
%   % dvi code
% \fi
% The latest version of ifpdf.sty can be obtained from:
% http://www.ctan.org/tex-archive/macros/latex/contrib/oberdiek/
% Also, note that IEEEtran.cls V1.7 and later provides a builtin
% \ifCLASSINFOpdf conditional that works the same way.
% When switching from latex to pdflatex and vice-versa, the compiler may
% have to be run twice to clear warning/error messages.






% *** CITATION PACKAGES ***
%
%\usepackage{cite}
% cite.sty was written by Donald Arseneau
% V1.6 and later of IEEEtran pre-defines the format of the cite.sty package
% \cite{} output to follow that of IEEE. Loading the cite package will
% result in citation numbers being automatically sorted and properly
% "compressed/ranged". e.g., [1], [9], [2], [7], [5], [6] without using
% cite.sty will become [1], [2], [5]--[7], [9] using cite.sty. cite.sty's
% \cite will automatically add leading space, if needed. Use cite.sty's
% noadjust option (cite.sty V3.8 and later) if you want to turn this off.
% cite.sty is already installed on most LaTeX systems. Be sure and use
% version 4.0 (2003-05-27) and later if using hyperref.sty. cite.sty does
% not currently provide for hyperlinked citations.
% The latest version can be obtained at:
% http://www.ctan.org/tex-archive/macros/latex/contrib/cite/
% The documentation is contained in the cite.sty file itself.






% *** GRAPHICS RELATED PACKAGES ***
%
\ifCLASSINFOpdf
  % \usepackage[pdftex]{graphicx}
  % declare the path(s) where your graphic files are
  % \graphicspath{{../pdf/}{../jpeg/}}
  % and their extensions so you won't have to specify these with
  % every instance of \includegraphics
  % \DeclareGraphicsExtensions{.pdf,.jpeg,.png}
\else
  % or other class option (dvipsone, dvipdf, if not using dvips). graphicx
  % will default to the driver specified in the system graphics.cfg if no
  % driver is specified.
  % \usepackage[dvips]{graphicx}
  % declare the path(s) where your graphic files are
  % \graphicspath{{../eps/}}
  % and their extensions so you won't have to specify these with
  % every instance of \includegraphics
  % \DeclareGraphicsExtensions{.eps}
\fi
% graphicx was written by David Carlisle and Sebastian Rahtz. It is
% required if you want graphics, photos, etc. graphicx.sty is already
% installed on most LaTeX systems. The latest version and documentation can
% be obtained at:
% http://www.ctan.org/tex-archive/macros/latex/required/graphics/
% Another good source of documentation is "Using Imported Graphics in
% LaTeX2e" by Keith Reckdahl which can be found as epslatex.ps or
% epslatex.pdf at: http://www.ctan.org/tex-archive/info/
%
% latex, and pdflatex in dvi mode, support graphics in encapsulated
% postscript (.eps) format. pdflatex in pdf mode supports graphics
% in .pdf, .jpeg, .png and .mps (metapost) formats. Users should ensure
% that all non-photo figures use a vector format (.eps, .pdf, .mps) and
% not a bitmapped formats (.jpeg, .png). IEEE frowns on bitmapped formats
% which can result in "jaggedy"/blurry rendering of lines and letters as
% well as large increases in file sizes.
%
% You can find documentation about the pdfTeX application at:
% http://www.tug.org/applications/pdftex





% *** MATH PACKAGES ***
%
%\usepackage[cmex10]{amsmath}
% A popular package from the American Mathematical Society that provides
% many useful and powerful commands for dealing with mathematics. If using
% it, be sure to load this package with the cmex10 option to ensure that
% only type 1 fonts will utilized at all point sizes. Without this option,
% it is possible that some math symbols, particularly those within
% footnotes, will be rendered in bitmap form which will result in a
% document that can not be IEEE Xplore compliant!
%
% Also, note that the amsmath package sets \interdisplaylinepenalty to 10000
% thus preventing page breaks from occurring within multiline equations. Use:
%\interdisplaylinepenalty=2500
% after loading amsmath to restore such page breaks as IEEEtran.cls normally
% does. amsmath.sty is already installed on most LaTeX systems. The latest
% version and documentation can be obtained at:
% http://www.ctan.org/tex-archive/macros/latex/required/amslatex/math/





% *** SPECIALIZED LIST PACKAGES ***
%
%\usepackage{algorithmic}
% algorithmic.sty was written by Peter Williams and Rogerio Brito.
% This package provides an algorithmic environment fo describing algorithms.
% You can use the algorithmic environment in-text or within a figure
% environment to provide for a floating algorithm. Do NOT use the algorithm
% floating environment provided by algorithm.sty (by the same authors) or
% algorithm2e.sty (by Christophe Fiorio) as IEEE does not use dedicated
% algorithm float types and packages that provide these will not provide
% correct IEEE style captions. The latest version and documentation of
% algorithmic.sty can be obtained at:
% http://www.ctan.org/tex-archive/macros/latex/contrib/algorithms/
% There is also a support site at:
% http://algorithms.berlios.de/index.html
% Also of interest may be the (relatively newer and more customizable)
% algorithmicx.sty package by Szasz Janos:
% http://www.ctan.org/tex-archive/macros/latex/contrib/algorithmicx/




% *** ALIGNMENT PACKAGES ***
%
\usepackage{array}
% Frank Mittelbach's and David Carlisle's array.sty patches and improves
% the standard LaTeX2e array and tabular environments to provide better
% appearance and additional user controls. As the default LaTeX2e table
% generation code is lacking to the point of almost being broken with
% respect to the quality of the end results, all users are strongly
% advised to use an enhanced (at the very least that provided by array.sty)
% set of table tools. array.sty is already installed on most systems. The
% latest version and documentation can be obtained at:
% http://www.ctan.org/tex-archive/macros/latex/required/tools/


%\usepackage{mdwmath}
%\usepackage{mdwtab}
% Also highly recommended is Mark Wooding's extremely powerful MDW tools,
% especially mdwmath.sty and mdwtab.sty which are used to format equations
% and tables, respectively. The MDWtools set is already installed on most
% LaTeX systems. The lastest version and documentation is available at:
% http://www.ctan.org/tex-archive/macros/latex/contrib/mdwtools/


% IEEEtran contains the IEEEeqnarray family of commands that can be used to
% generate multiline equations as well as matrices, tables, etc., of high
% quality.


%\usepackage{eqparbox}
% Also of notable interest is Scott Pakin's eqparbox package for creating
% (automatically sized) equal width boxes - aka "natural width parboxes".
% Available at:
% http://www.ctan.org/tex-archive/macros/latex/contrib/eqparbox/





% *** SUBFIGURE PACKAGES ***
%\usepackage[tight,footnotesize]{subfigure}
% subfigure.sty was written by Steven Douglas Cochran. This package makes it
% easy to put subfigures in your figures. e.g., "Figure 1a and 1b". For IEEE
% work, it is a good idea to load it with the tight package option to reduce
% the amount of white space around the subfigures. subfigure.sty is already
% installed on most LaTeX systems. The latest version and documentation can
% be obtained at:
% http://www.ctan.org/tex-archive/obsolete/macros/latex/contrib/subfigure/
% subfigure.sty has been superceeded by subfig.sty.



%\usepackage[caption=false]{caption}
\usepackage[font=footnotesize]{subfig}
% subfig.sty, also written by Steven Douglas Cochran, is the modern
% replacement for subfigure.sty. However, subfig.sty requires and
% automatically loads Axel Sommerfeldt's caption.sty which will override
% IEEEtran.cls handling of captions and this will result in nonIEEE style
% figure/table captions. To prevent this problem, be sure and preload
% caption.sty with its "caption=false" package option. This is will preserve
% IEEEtran.cls handing of captions. Version 1.3 (2005/06/28) and later
% (recommended due to many improvements over 1.2) of subfig.sty supports
% the caption=false option directly:
%\usepackage[caption=false,font=footnotesize]{subfig}
%
% The latest version and documentation can be obtained at:
% http://www.ctan.org/tex-archive/macros/latex/contrib/subfig/
% The latest version and documentation of caption.sty can be obtained at:
% http://www.ctan.org/tex-archive/macros/latex/contrib/caption/




% *** FLOAT PACKAGES ***
%
\usepackage{fixltx2e}
% fixltx2e, the successor to the earlier fix2col.sty, was written by
% Frank Mittelbach and David Carlisle. This package corrects a few problems
% in the LaTeX2e kernel, the most notable of which is that in current
% LaTeX2e releases, the ordering of single and double column floats is not
% guaranteed to be preserved. Thus, an unpatched LaTeX2e can allow a
% single column figure to be placed prior to an earlier double column
% figure. The latest version and documentation can be found at:
% http://www.ctan.org/tex-archive/macros/latex/base/



\usepackage{stfloats}
% stfloats.sty was written by Sigitas Tolusis. This package gives LaTeX2e
% the ability to do double column floats at the bottom of the page as well
% as the top. (e.g., "\begin{figure*}[!b]" is not normally possible in
% LaTeX2e). It also provides a command:
%\fnbelowfloat
% to enable the placement of footnotes below bottom floats (the standard
% LaTeX2e kernel puts them above bottom floats). This is an invasive package
% which rewrites many portions of the LaTeX2e float routines. It may not work
% with other packages that modify the LaTeX2e float routines. The latest
% version and documentation can be obtained at:
% http://www.ctan.org/tex-archive/macros/latex/contrib/sttools/
% Documentation is contained in the stfloats.sty comments as well as in the
% presfull.pdf file. Do not use the stfloats baselinefloat ability as IEEE
% does not allow \baselineskip to stretch. Authors submitting work to the
% IEEE should note that IEEE rarely uses double column equations and
% that authors should try to avoid such use. Do not be tempted to use the
% cuted.sty or midfloat.sty packages (also by Sigitas Tolusis) as IEEE does
% not format its papers in such ways.





% *** PDF, URL AND HYPERLINK PACKAGES ***
%
%\usepackage{url}
% url.sty was written by Donald Arseneau. It provides better support for
% handling and breaking URLs. url.sty is already installed on most LaTeX
% systems. The latest version can be obtained at:
% http://www.ctan.org/tex-archive/macros/latex/contrib/misc/
% Read the url.sty source comments for usage information. Basically,
% \url{my_url_here}.





% *** Do not adjust lengths that control margins, column widths, etc. ***
% *** Do not use packages that alter fonts (such as pslatex).         ***
% There should be no need to do such things with IEEEtran.cls V1.6 and later.
% (Unless specifically asked to do so by the journal or conference you plan
% to submit to, of course. )


% correct bad hyphenation here
\hyphenation{op-tical net-works semi-conduc-tor}


\begin{document}
%
% paper title
% can use linebreaks \\ within to get better formatting as desired
\title{Solar Pro}


% author names and affiliations
% use a multiple column layout for up to three different
% affiliations
\author{\IEEEauthorblockN{Karthik Sukumar}
\IEEEauthorblockA{Electrical and Computer Engineering\\
Technical University of Munich\\
Munich, Germany\\
karthik.sukumar@tum.de}
\and
\IEEEauthorblockN{Johannes Machleid}
\IEEEauthorblockA{Electrical and Computer Engineering\\
Technical University of Munich\\
Munich, Germany\\
johannes.machleid@tum.de
}


% \thanks{*This paper is a reinterpretation of the paper \emph{J. Caesar. ``Digital sundials and broadband technology,'' in Proc.
% IOOC-ECOC, 19XX, pp. 557-998}. It was presented on December 24, 2006 (Paper submission deadline) as a part of MSCE Seminar (MSCE course TUM), under the supervision of M. Sc. (or Dipl. Ing.)
% Aurelia Cotta (aurelia.cotta@roma.it).}
}

% conference papers do not typically use \thanks and this command
% is locked out in conference mode. If really needed, such as for
% the acknowledgment of grants, issue a \IEEEoverridecommandlockouts
% after \documentclass

% for over three affiliations, or if they all won't fit within the width
% of the page, use this alternative format:
%
%\author{\IEEEauthorblockN{Michael Shell\IEEEauthorrefmark{1},
%Homer Simpson\IEEEauthorrefmark{2},
%James Kirk\IEEEauthorrefmark{3},
%Montgomery Scott\IEEEauthorrefmark{3} and
%Eldon Tyrell\IEEEauthorrefmark{4}}
%\IEEEauthorblockA{\IEEEauthorrefmark{1}School of Electrical and Computer Engineering\\
%Georgia Institute of Technology,
%Atlanta, Georgia 30332--0250\\ Email: see http://www.michaelshell.org/contact.html}
%\IEEEauthorblockA{\IEEEauthorrefmark{2}Twentieth Century Fox, Springfield, USA\\
%Email: homer@thesimpsons.com}
%\IEEEauthorblockA{\IEEEauthorrefmark{3}Starfleet Academy, San Francisco, California 96678-2391\\
%Telephone: (800) 555--1212, Fax: (888) 555--1212}
%\IEEEauthorblockA{\IEEEauthorrefmark{4}Tyrell Inc., 123 Replicant Street, Los Angeles, California 90210--4321}}




% use for special paper notices
%\IEEEspecialpapernotice{(Invited Paper)}




% make the title area
\maketitle


\begin{abstract}
%\boldmath
    The source of future energy production has to be and is undoubtedly renewable and environmentally friendly. Solar power is one of the most readily available energy sources in almost all parts of the world. Although solar power is ubiquitous, certain physical and technological limitations allow for a maximum efficiency factor of 37\% (and that's for commercially avaiable high end solar cells). This implies that only 37\% of the sun's energy captured by the solar cell can be converted to useful electrical energy. This paper focuses on using Wireless Sensor Networks (WSN) to utilise the maximum possible energy of the solar cells without any further losses by aligning the solar panel orthogonal to the sunrays depending on the daytime.
\end{abstract}
% IEEEtran.cls defaults to using nonbold math in the Abstract.
% This preserves the distinction between vectors and scalars. However,
% if the conference you are submitting to favors bold math in the abstract,
% then you can use LaTeX's standard command \boldmath at the very start
% of the abstract to achieve this. Many IEEE journals/conferences frown on
% math in the abstract anyway.

% no keywords




% For peer review papers, you can put extra information on the cover
% page as needed:
% \ifCLASSOPTIONpeerreview
% \begin{center} \bfseries EDICS Category: 3-BBND \end{center}
% \fi
%
% For peerreview papers, this IEEEtran command inserts a page break and
% creates the second title. It will be ignored for other modes.
\IEEEpeerreviewmaketitle



\section{Introduction}
% no \IEEEPARstart

The energy generated by the solar panels is highly dependent on the angle of incidence of the sunrays on the panel.

\begin{figure}[htbp]
	\includegraphics[width=2.5in]{img/SunsPath1.png}
	\centering
    \captionsetup{justification=centering}
	\caption{Sun's path in winter and summer \cite{b1}.}
	\label{fig:SunsPath1}
\end{figure}

The sun's path in the horizon is dynamic and dependent on the time of the year. In the northern hemisphere, the sun is higher in the sky during summer and lower in the winter. In the summer the sun rises in the north east and sets in the north west. Whereas in winter, the sun rises in the south east and sets in the south west as it seen in Fig.~\ref{fig:SunsPath1}. As shown in Fig.~\ref{fig:SunsPath2}, consequent of the sun's path in the sky, the angle of incidence of the sun's rays on the panel changes not only during the day but as well as during the months of the year. In order to track the sun's rays, so that the solar panel is always perpendicular to it, a dynamic system that adjusts to the east-west changes in the sun's angle during the different times of the day as well the north-south tracking during the changes in the months is needed.

\begin{figure}[htbp]
    \includegraphics[width=2.5in]{img/SunsPath2.png}
    \centering
    \captionsetup{justification=centering}
    \caption{Angle of incidence of the sun's rays on the panel \cite{b2}.}
    \label{fig:SunsPath2}
\end{figure}


\begin{figure}[htbp]
	\includegraphics[width=70mm]{img/ComparisonTrackingVsNoTracking.png}
	\centering
	\captionsetup{justification=centering}
	\caption{Comparison of energy produced by a tracking vs non tracking system \cite{b3}.}
	\label{fig:ComparisonTvsNT}
\end{figure}


\section{Application Set up}
Fig.~\ref{fig:ComparisonTvsNT} is a comparison of the percentage of total power generated between a tracking and a non-tracking panel. It shows that there are definitely efficiency gains to be made from employing a tracking panel. Our proof of concept therefore showcases the potential to deploy a wireless sensor network to not only control the tracking of solar panels but log data at the same time. In order to achieve this, we employ three different types of sensors and eight wireless motes, of which one is set up as the base station connected to a desktop computer and the seven others deployed around the base station to function as sensor and panel control motes.\\
In this proof of concept as only four digital servo motors are available, not every mote in the field is able to align the solar panel according to the sun's rays. Never the less sensor values can still be measured and transmitted. A possible setup with a multihop communication towards the base station is depicted in Fig.~\ref{fig:ApplicationSetup}.

\begin{figure}[!t]
    \includegraphics[width=3in]{img/Application_Setup.png}
    \centering
    \captionsetup{justification=centering}
    \caption{Possible application setup with seven sensor motes and a base station ensuring multihop communication.}
    \label{fig:ApplicationSetup}
\end{figure}

\subsection{Sensors}
Described below are the different types of sensors used in our application along with their uses:

\begin{itemize}
    \item Digital Servo Motor - HS-422\cite{servoMotor}\newline
        A servo motor is a cheap rotary actuator which is connected to the mote via three wires: Ground (GND), Voltage (VDD) and Signal. The servo angle can be set via the signal wire by using a PWM-Signal. Most servos expect to see a pulse every 20ms~\cite{howServoworks}. The correspondence between pulsewidth and servo angle is depicted in formula~\eqref{eq:Servo_PWM}. The servos will be used to change the angle of solar panels.
\begin{equation}\label{eq:Servo_PWM}
pw = 1.0ms + 1.0ms \cdot \left(\frac{\alpha}{\alpha_{max}}\right)
\end{equation}

    \item Light Sensors\cite{LightSensor}\newline
        The light sensors are used in this proof of concept to measure the luminosity and thus simulate the power output of the solar panel, since no real solar panels are available. As a result, the relative ADC 12-bit values of the sensors are of interest.

    \item Wind Speed Sensor\cite{WindSensor}\newline
        The wind speed sensor, also called anemometer, is directly connected to the base station. It measures the wind speed at all times. In the case of excessive wind speeds, the panels will have to be stowed at a safe angle.\newline
        The anemometer works with a simple reed contact and a magnet attached to the rotating axis~\cite{anemometer}. Every time the magnet passes the reed contact, a digital HIGH is sent to the base station, which is processed by the base station using interrupts. The measured interval $t_{measure}$ together with the interrupt count $n_{ticks}$ delivers the actual wind speed \textit{ws} in km/h after formula~\eqref{eq:WindSpeed}.
        \begin{equation}\label{eq:WindSpeed}
            ws = \frac{n_{ticks}}{t_{measure}} \cdot 2.4 \dfrac{km/h}{tick}
        \end{equation}

\end{itemize}

\subsection{Motes}
The Zolertia RE-mote platform\cite{REmote} is a wireless module designed to be small, consume very little power, stay affordable and be easy to deploy in large quantities. In general, these type of devices are known as \textit{motes}. These motes are the brain of the wireless sensor network and are running Contiki as an operating system. Eight motes are used in total out of which one mote functions as a base station and seven motes act as panel motes.

\subsubsection{Base Station}
    The base station is connected to a desktop computer via a serial link and feeds the GUI with information. It runs a different code compared to the panel motes as it has to account for interfacing with the computer and collect data from the wind speed sensor at the same time.
\subsubsection{Panel Motes}
    The panel motes run a more simplified subset of the base station code. They only react to unicast and broadcast messages and they respond based on the packet type.

\section{Network description and Routing Technique}
In this proof of concept we are demonstrating an WSN which provides information on the status of the solar panel along with the ability to control the angle of the panel in east-west axis. In a real world scenario it is intended that the mote will be powered by the solar panel and be able to draw minimal required power for its functioning.
\subsection{Network}
Considering our application, robustness and reliability were important factors for our design. Hence the decision was made to have a centralised base station polling the motes regularly. Having a master that queries, eliminates the need for system wide synchronisation (which are potentially required in a distributed system) and makes the implementation simpler. This also has the added advantage of avoiding collisions and random backoffs in unicast modes as at any point in time there is only ever one node using the channel. In this heterogenous network the routes are first discovered during the network discovery phase and the routing tables are exchanged as described in \ref{NetDisc}.

\subsection{Base Station State Machine}
Fig.~\ref{fig:StateMachine} shows the five states of operation and the transition conditions. The states are:
\begin{itemize}
    \item IDLE
    \item NETWORKDISCOVERY
    \item PATHMODE
    \item UNICASTMODE
    \item EMERGENCY
\end{itemize}

\subsubsection{IDLE} \label{IDLEMODE}
After bootup the device enters the IDLE state. In this state there are no broadcast or unicast messages that are sent out by the base station.

\subsubsection{NETWORKDISCOVERY} \label{NetDiscMode}
This state is entered upon a user button press on the REmote or a trigger from the GUI. This event driven mode is used to obtain the network graph via a series of broadcast messages. A more detailed description can be found in~\ref{NetDisc}

\subsubsection{PATHMODE} \label{PathMode}
The base station transitions into PATHMODE state after the NETWORKDISCOVERY mode. This is a query-driven data reporting method, where the base station polls each node for their hop traces to the base station. This is required for the GUI to be able to display the topology.

\subsubsection{UNICASTMODE} \label{UNICASTMODE}
In this state, the nodes are polled in a round-robin manner and when the base station queries the information, it sends the servo angle data along with the query packet. The nodes respond by filling out the fields with their respective sensor values. Packet fields include battery voltage, temperature, light sensor value and servo angle. This state also follows a query driven data reporting method.

\subsubsection{EMERGENCY} \label{EMERGENCY}
This state is only entered upon when the base station records an over threshold wind speed value. This event driven mode sends out a broadcast message to all nodes around it to stow their panels at a predefined safe angle.

\begin{figure}[!t]
    %\includegraphics[width=2.5in]{img/Application_Flow_Graph.jpg}
    \includegraphics[width=90mm]{img/Application_Flow_Graph.jpg}\centering
    \captionsetup{justification=centering}
    \caption{Base station state machine.}
    \label{fig:StateMachine}
\end{figure}

\subsection{Routing Cost Metric} \label{Cost}
A very important consideration is the cost for a route. There are many factors that can determine the calculation of a route cost. It could include battery state, transmission power, distance hop count and RSSI. In our application we have directly equated hop count as a distance cost metric.

\subsection{Network Discovery} \label{NetDisc}
This is the most critical phase of the state machine. The network discovery here is started by a user button press for ease of demonstration of our proof of concept. Furthermore it can also be triggered from the GUI as explained in section~\ref{GUI}. Any mote can initiate a network discovery which triggers a controlled flooding broadcast message. The node's complete routing table is sent as a payload to exchange with others. This method works in accordance to the distance vector routing and follows the Bellman-Ford shortest path tree algorithm. The routing itself is distributed where each node receives information from direct neighbours, performs calculation and distributes the result back to its neighbours, if there were changes to be made to the receiver's routing table. This means the network discovery is iterative and self terminating and also asynchronous.

\subsection{Fault Tolerance} \label{FaultTol}
The wireless medium is prone to introducing errors, particularly because of noise, interference and attenuation. The dynamic nature of the wireless sensor network in the field could mean that nodes can be damaged, non-functional or removed from the network. To cope with these scenarios the application has to be fault tolerant. An acknowledgement ensures detecting transmission errors. If the sending node does not receive an acknowledge message after a certain number of tries, it initiates a network discovery.


%\subsection{Packet Types} \label{PacketTypes}
%We have 2 different communication schemes which can be found below
%\begin{itemize}
%    \item Broadcast
%        \begin{itemize}
%            \item EMERGENCY - This broadcast packet is sent out in the emergency phase which stows the panels at a predefined safe angle
%
%            \item NETDISC - This packet type is sent out during the network discovery phase which contains sender's routing tables
%        \end{itemize}
%    \item Unicast
%        \begin{itemize}
%            \item PATH - This unicast packet type is sent during the PATHMODE of operation as described in \ref{PathMode}
%            \item ACK - This unicast packet type is an ackledgement responce sent during the PATHMODE. This contains the hop history of from the base station to the target node
%            \item UNICAST - This unicast packet type is sent during the UNICASTMODE of operation as described in \ref{UNICASTMODE}. In this mode the nodes are polled in round-robin fashion at regular intervals and this goes on uninterrupted unless an Emergency is triggered
%        \end{itemize}
%\end{itemize}

\section{Graphical User Interface (GUI)} \label{GUI}
The graphical user interface (GUI) is a desktop application programmed with the aid of the QT Creator and focuses on displaying information about the motes operating in the field and allows the user to interact with a desired mote. The computer running the GUI is directly connected to the base station mote via a serial link.\\
The GUI basically consists of the two tabs "General" and "Connections" as shown in Fig.~\ref{GUI_general} and Fig.~\ref{GUI_connections}.\\
The "General" tab shows a network graph of the wireless sensor network and the system time. Communicating sensor motes are depicted with a yellow node and a node ID inside, whereas the base station is shown centered as green node.\\
The double sided arrows depict the wireless communication route and therefore the network topology. Only the shortest paths of the nodes towards the base station as sink in the network is depicted on the graph.\\
The two buttons "Network Discovery" and "Emergency" in the bottom left corner enable the user to trigger these two states~\ref{NetDiscMode} and~\ref{EMERGENCY} manually.\\
Two info boxes aligned to the right side of the window show the sensor information of the motes. The upper info box shows the values of the anemometer connected to the base station, such as the actual wind speed, the average windspeed, the maximum windspeed and the emergency threshold, at which the solar panels are aligned to a predefined safety angle. The GUI also allows to choose a user defined emergency threshold, which is set by pressing the "Set" button next to the number field. The default value is 16 km/h.\\
The lower info box shows the sensor information of the selected mote. A mote is selected by clicking on a mote in the network graph. The particular information such as the Node ID, the temperature, voltage, luminosity and panel angle are presented. For maintenance reasons it is also possible to operate the selected mote by clicking on the "Manual" button. It is then possible to enter an angle between 0 and 180 degrees to manually align the selected solar panel by clicking the "Set" button next to the number field.\\
The second tab of the GUI allows the user to connect and disconnect the serial link to the base station via selecting the respective port and clicking the buttons "Open" to connect or "Close" to disconnect. The textbox aligned to the bottom of the tab shows possible debug information from the base station mote.


\begin{figure*}[!ht]
\centering
\subfloat[GUI General Tab]{\includegraphics[width=3.5in]{img/GUI_screenshot_general.png}
\label{GUI_general}}
\hfil
\subfloat[GUI Connections Tab]{\includegraphics[width=3.5in]{img/GUI_screenshot_connections.png}
\label{GUI_connections}}
\caption{Graphical user interface of the Solar Pro application.}
\label{GUIFig}
\end{figure*}


\section{Application specific features}
The radio duty cycle (RDC) layer is not modified and runs the Contiki standard configuration "ContikiMAC" according to~\cite{dunkels2011contikimac}. The contiki MAC provides energy savings due to duty cycled wake up times. The carrier sense multiple access (CSMA) driver provides reliability by sensing busy channel medium and backing off by a random amount of time before transmitting the packet. This avoids collisions during the broadcasting modes although it might increase the latency.

\section{Future work and Optimisations}
Our implementation does not focus on low latency so this could be a possible improvement. One way of improving latency can be achieved by using hash tables instead of array based routing tables. This reduces the complexity from O(n) to a constant time. Considering that for every UNICAST transmission routing table walk is necessary, this could save time and power. Further ideas to advance the application are a sun tracking algorithm which steers the solar panel according to sensor values. Another useful feature would be to integrate data visualization of the sensor values as a graph showing the history of e.g. the last 24 hours.

% An example of a floating figure using the graphicx package.
% Note that \label must occur AFTER (or within) \caption.
% For figures, \caption should occur after the \includegraphics.
% Note that IEEEtran v1.7 and later has special internal code that
% is designed to preserve the operation of \label within \caption
% even when the captionsoff option is in effect. However, because
% of issues like this, it may be the safest practice to put all your
% \label just after \caption rather than within \caption{}.
%
% Reminder: the "draftcls" or "draftclsnofoot", not "draft", class
% option should be used if it is desired that the figures are to be
% displayed while in draft mode.
%
%\begin{figure}[!t]
%\centering
%\includegraphics[width=2.5in]{myfigure}
% where an .eps filename suffix will be assumed under latex,
% and a .pdf suffix will be assumed for pdflatex; or what has been declared
% via \DeclareGraphicsExtensions.
%\caption{Simulation Results}
%\label{fig_sim}
%\end{figure}

% Note that IEEE typically puts floats only at the top, even when this
% results in a large percentage of a column being occupied by floats.


% An example of a double column floating figure using two subfigures.
% (The subfig.sty package must be loaded for this to work.)
% The subfigure \label commands are set within each subfloat command, the
% \label for the overall figure must come after \caption.
% \hfil must be used as a separator to get equal spacing.
% The subfigure.sty package works much the same way, except \subfigure is
% used instead of \subfloat.
%
%\begin{figure*}[!t]
%\centerline{\subfloat[Case I]\includegraphics[width=2.5in]{subfigcase1}%
%\label{fig_first_case}}
%\hfil
%\subfloat[Case II]{\includegraphics[width=2.5in]{subfigcase2}%
%\label{fig_second_case}}}
%\caption{Simulation results}
%\label{fig_sim}
%\end{figure*}
%
% Note that often IEEE papers with subfigures do not employ subfigure
% captions (using the optional argument to \subfloat), but instead will
% reference/describe all of them (a), (b), etc., within the main caption.


% An example of a floating table. Note that, for IEEE style tables, the
% \caption command should come BEFORE the table. Table text will default to
% \footnotesize as IEEE normally uses this smaller font for tables.
% The \label must come after \caption as always.
%
%\begin{table}[!t]
%% increase table row spacing, adjust to taste
%\renewcommand{\arraystretch}{1.3}
% if using array.sty, it might be a good idea to tweak the value of
% \extrarowheight as needed to properly center the text within the cells
%\caption{An Example of a Table}
%\label{table_example}
%\centering
%% Some packages, such as MDW tools, offer better commands for making tables
%% than the plain LaTeX2e tabular which is used here.
%\begin{tabular}{|c||c|}
%\hline
%One & Two\\
%\hline
%Three & Four\\
%\hline
%\end{tabular}
%\end{table}


% Note that IEEE does not put floats in the very first column - or typically
% anywhere on the first page for that matter. Also, in-text middle ("here")
% positioning is not used. Most IEEE journals/conferences use top floats
% exclusively. Note that, LaTeX2e, unlike IEEE journals/conferences, places
% footnotes above bottom floats. This can be corrected via the \fnbelowfloat
% command of the stfloats package.

\section{Conclusion}
The power efficiency of solar panels highly depends on the inclination angle of the sunrays. By aligning the solar panel orthogonally to the sunrays depending on the daytime, it is possible to increase its energy efficiency. In this application a wireless sensor network has been employed to control the solar panel angle via a digital servo and measure the amount of light hitting the panel with a light sensor. Furthermore the temperature of the controlling mote and its battery voltage is read out.\\
The application setup consists of seven sensor motes, of which four motes operate a digital servo. One mote functions as base station and delivers the sensor information to a desktop application with a GUI.\\
The network is set up by first initialising a network discovery routine. All motes broadcast and collect information about their neighbors to set up their very own distance vector table. After completion, the base station reaches out to each mote with a unicast request to collect the shortest path through their respective acknowledge messages.\\
With this information the base station is able to poll the sensor informations of each node by cycling through the network.\\
In case of lost packets or transmission errors, the network reorganises itself after a certain number of retransmissions.\\
The network graph and sensor information are visually presented in the graphical user interface, which also allows to operate the servo position manually, set a wind speed emergency threshold or simply trigger an emergency case manually. It is also possible to manually restart the network discovery.

% conference papers do not normally have an appendix


% use section* for acknowledgement
%\section*{Acknowledgment}


%The authors would like to thank...





% trigger a \newpage just before the given reference
% number - used to balance the columns on the last page
% adjust value as needed - may need to be readjusted if
% the document is modified later
%\IEEEtriggeratref{8}
% The "triggered" command can be changed if desired:
%\IEEEtriggercmd{\enlargethispage{-5in}}

% references section

% can use a bibliography generated by BibTeX as a .bbl file
% BibTeX documentation can be easily obtained at:
% http://www.ctan.org/tex-archive/biblio/bibtex/contrib/doc/
% The IEEEtran BibTeX style support page is at:
% http://www.michaelshell.org/tex/ieeetran/bibtex/
%\bibliographystyle{IEEEtran}
% argument is your BibTeX string definitions and bibliography database(s)
%\bibliography{IEEEabrv,../bib/paper}
%
% <OR> manually copy in the resultant .bbl file
% set second argument of \begin to the number of references
% (used to reserve space for the reference number labels box)
%\newpage
\bibliography{IEEEabrv,sources}
\bibliographystyle{IEEEtran}


% that's all folks
\end{document}
